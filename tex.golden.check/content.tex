

\end{multicols}
\chapter{מילון הראיה}\addPolythumb{מילון הראיה}
\fancyhead[CE,CO]{מילון הראיה}
\begin{multicols}{2}


\end{multicols}
\subsubsection{א}\replacePolythumb{א}
\fancyhead[CO]{אות א}
\begin{multicols}{2}
\משנה{"יורו" שנאמר ביחס ל"בין קדש לח}\הגדרה{ֹ}\משנה{ל" לעומת "יוֹדִעֻם" שנאמר ביחס ל"בין טמא לטהור"}\footnote{ יחזקאל מד כג.}\משנה{ }\צהגדרה{- לפי שיש הבדל במצב}\\

\ערך{שם טוב}\footnote{ אבות ב ז, ד יג.}\תקלה{ }\הגדרה{- }\משנה{תואר שם טוב }\הגדרה{- ההשלמה שפועל על זולתו כפי ערכו }\מקור{[ע"א א ב נג]}\צהגדרה{. }\\

\ערך{שכינה }\הגדרה{- "}\משנה{ענן השכינה"}\footnote{ תנחומא במדבר יב.}\הגדרה{ - }\הגדרה{הוד°}\הגדרה{ }\הגדרה{הכבוד־העליון°}\הגדרה{, המעולף בענני ערפל לרדת ללב בני האדם להחיותם חיי }\הגדרה{עד°}\הגדרה{ }\מקור{[ע"ר א פח]}\צהגדרה{. }\\

\ערך{אֶבֶן דִי לָא בִיְדַיִן}\footnote{ דניאל ב לד "חָזֵה הֲוַיְתָ עַד דִּי הִתְגְּזֶרֶת אֶבֶן דִּי לָא בִידַיִן וּמְחָת לְצַלְמָא וגו'". ע"ע פנק' ג קנ.}\תקלה{ }\הגדרה{- }\משנה{די לא בידין }\הגדרה{- }\מעוין{◊}\הגדרה{ אם היה רצון }\הגדרה{השי"ת°}\הגדרה{ לעכב את ישראל עד שישלימו הם בעצמם במעשיהם הטובים את הכשרם אל }\הגדרה{הגאולה°}\הגדרה{, היתה האבן נקראת }\משנה{אבן די בידין}\הגדרה{, שבידים שהיו ישראל עושים את תיקוני שלמותם, בזה }\משנה{ימחו לצלמא°}\הגדרה{, שהוא כח הרע והקלקול שבמציאות שאינו נותן מקום לגדולת ישראל. אבל כיון שרצון השי"ת הוא רק שיהיו ישראל מוכנים קצת ואז השי"ת ינטלם וינשאם אל קדושת המעלה הראויה להם אחרי }\הגדרה{שימורק°}\הגדרה{ }\הגדרה{עונם°}\הגדרה{ ויקבלו איזו שלמות להכנה של מעלתם, אע"פ שלא יהיו עדיין מוכנים לגמרי, א"כ תהיה השבירה של הצלם ע"י }\משנה{אבן די לא בידין }\הגדרה{שלא עשו ישראל בידיהם זאת המעלה }\מקור{[מ"ש רעז־ח]}\צהגדרה{.}\\

\ערך{זיו אלהי }\הגדרה{- }\משנה{מעוז הזיו האלהי שממעל לכל גבולי־עולמים }\הגדרה{- }\הגדרה{הגודל־העליון°}\הגדרה{ }\מקור{[ע"ר א כ, ועפ"י שם ב עד]}\צהגדרה{.}\\\הגדרה{}\הגדרה{שכינתא°}\הגדרה{}\מקור{ [א"ק ג יט]}\צהגדרה{. }\\\ערך{זיו}\footnote{ זיו - בדעת תבונות, עמ' עה (מהד' ר"ח פרידלנדר) כותב הרמח"ל כהקבלה לסדרי הבריאה של הקב"ה ודרכיה בבריאה עצמה: "יש דבר אחד נמצא מהתחברות הנשמה והגוף - הוא זיו הפנים... ואין הזיו הזה נמצא לא לנשמה בפני עצמה, ולא לגוף בפני עצמו, אבל הוא הדבר הנולד מחיבור הנשמה והגוף ביחד". ובדרך מצותיך נא. "זיו הנפש הוא ענין חיוני מתפשט בגוף שהוא מקבל חיות זה וחי ממנו. וכך מהותו ועצמותו המחוייב המציאות, שהוא לבדו הוא, ואין זולתו, והוא חיי החיים, כשימשיך ויאיר ממנו ויגלה הארה בבחי' זיו ענינה הוא המשכת חיים בבחי' א"ס וז"ש הודו על ארץ ושמים כו' ופי' הוד וזיו". וע' בנספחות, מדור מחקרים, אור, זיו, ברק. ואור, זוהר, זיו.}\ערך{ אור אלהים }\הגדרה{- }\משנה{זיו החיים }\הגדרה{- }\הגדרה{שכינת־אל°}\הגדרה{. }\הגדרה{אור־אלהים°}\הגדרה{, הממלא את }\הגדרה{העולמים°}\הגדרה{ כולם}\\

\ערך{זמן }\הגדרה{- }\משנה{(לעומת נצח°) }\הגדרה{- ההוה התדירי }\מקור{[ע"ר א סד]}\צהגדרה{. }\\

\ערך{גשמים}\footnote{ אבות ה משנה ה.}\ערך{, הבאים מיסוד המים }\הגדרה{- מורים על עצם ההויה הגופנית, שֶׂכֶל הגוף }\מקור{[ע"ר ב קעו]}\צהגדרה{. }\\

\מעוין{◊ }\משנה{תאר מלך}\הגדרה{ ראוי להאמר מצד הכרתינו את }\הגדרה{דעת־האלהים°}\הגדרה{ מצד הנבראים כפי ערכינו }\מקור{[ע"א ב ח ו]}\צהגדרה{. }\\

\הגדרה{ע"ע אור עליון. ע"ע אור ד'.}\footnote{ אור אלהי, אור אלהים, אור ד', אור עליון - בין מושגים אלה התקשתי למצוא הבדל, מכל מקום חולקו ההגדרות למחלקות שונות על פי המונחים השונים.}\הגדרה{ ע' במדור שמות כינויים ותארים אלהיים, אלהי, המקור האלהי. }\\

\ערך{תורה}\תקלה{60 }\הגדרה{- }\משנה{בעצמיות מהות הפנימית, הצורתית של התורה}\footnote{ עצמיות מהות הפנימי הצורתי של התורה - ע"ע לשב"ו, ס' הדע"ה, ח"ב, דרוש ד, ענף יב, סי' יב.}\הגדרה{ - עצמיות טבע נשמתם של ישראל }\מקור{[ע"א ד ט עו]}\צהגדרה{.}\\

\משנה{תורה }\צהגדרה{- }\משנה{מהותה המיוחדת של תורה לעומת הנבואה° }\צהגדרה{- השראת }\צהגדרה{השכינה°}\צהגדרה{ הכוללת, הקבועה, המוחלטת, "ואתה פה עמד עמדי", לעומת }\משנה{הנבואה }\צהגדרה{המכוונת אל עניני הדורות ותמורות מאורעותיהם "לפי צרך השעה, הדור והמעשה"}\footnote{ ע' רש"י ד"ה תורת משה חולין קלז.}\צהגדרה{ }\צמקור{[עפ"י ל"י א מז]. }\\

\תקלה{טבע }\הגדרה{- }\תקלה{(במובן המדעי הרגיל) }\צהגדרה{-}\תקלה{ חוקי ברזל, החוקיות }\תקלה{העיוורת°}\תקלה{ השולטת במציאות }\תקלה{(שהקדושה°}\תקלה{ לוחמת כנגדה) [עפ"י ב"ר שצז]. }\\\משנה{טבעיות }\צהגדרה{- היציבות המסודרת של התופעות }\צמקור{[ב"א 10]. }\\

\ערך{חילה }\הגדרה{- הידיעה במה שאין ראוי }\הגדרה{לצייר°}\הגדרה{, כמו חלל פנוי במשפט הכרתו }\הגדרה{מיראת־ד'°}\הגדרה{ }\מקור{[ע"ר א רנו]}\צהגדרה{. }\\

\ערך{זָכָר }\הגדרה{- היסוד העקרי שבתולדה והמשכת החיים, }\הגדרה{הסגולה°}\הגדרה{ המפעלית }\מקור{[עפ"י ע"ר א מ]}\צהגדרה{.}\\

\ערך{אור חַי־העולמים° }\הגדרה{- }\הגדרה{אור־עליון°}\הגדרה{, מקור מקורות, חיי החיים }\מקור{[קובץ ה צט]}\צהגדרה{.}\\

\מעוין{◊}\\

\ערך{אור הגלוי }\הגדרה{- }\משנה{האור }\הגדרה{}\הגדרה{הגלוי°}\הגדרה{ - האור הנראה של }\הגדרה{אור־התורה°}\הגדרה{ וחכמת ישראל כולה }\הגדרה{בקדושתה°}\הגדרה{ }\הגדרה{וטהרתה°}\הגדרה{, בבינתה והכרתה, בכבודה וישרותה, בעושר סעיפיה בעומק הגיונותיה ובאומץ מגמותיה. תלמודה של תורה בכל הרחבתה והסתעפו(יו)תיה, בדעת וכשרון, ברגש חי וקדוש, וברצון אדיר }\הגדרה{וחסון°}\הגדרה{ לחיות את אותם החיים הטהורים והקדושים אשר האור המלא הזה מתאר אותם לפנינו. }\הגדרה{(אור־הקודש־החבוי°}\הגדרה{) בהיותו מתקרב מאד אל מושגינו, אל צרכינו הזמניים, ואל מאויינו הלאומיים }\מקור{[מא"ה ג (מהדורת תשס"ד) קכג, קכה]}\צהגדרה{. }\\\הגדרה{ע"ע אור קודש חבוי. }\\

\תקלה{◊◊◊}\\

\תקלה{פרצופים: }\\

\משנה{פרצופים}\צהגדרה{ - }\משנה{עיקר ענינם }\צהגדרה{- }\צהגדרה{החיות האלהית העולמית והאחדות הכוללת }\צמקור{[ק"ה רכ]. }\\

\end{multicols}
\chapter{מונחי קבלה ונסתר}\addPolythumb{מונחי קבלה ונסתר}
\fancyhead[CE,CO]{מונחי קבלה ונסתר}
\begin{multicols}{2}


\תקלה{חילוני }\הגדרה{- }\תקלה{נשמה חילונית }\הגדרה{-}\תקלה{ זרם ההויה הבא מצד המשך ההויה שאנחנו קוראים }\תקלה{טבע°}\תקלה{, ההשגה של אוה"ע, הזרם הבא מהאמצעים }\תקלה{מהסיבות°}\תקלה{ }\תקלה{[נ"א ה 31].}\\\הגדרה{ע' במדור פסוקים ובטויי חז"ל, שכינה, גילוי שכינה.}\\

\תקלה{טבע }\הגדרה{-}\תקלה{ החוקיות הכוללת את האדם והעולם. חוקיות החורזת את כל סדרי המציאות }\צהגדרה{-}\תקלה{ האדם, הדורות, ההויה וההסתוריה, שמים וארץ. חוקיות המארגנת את כל הברואים והיצורים לחטיבה אחת [עפ"י ב"ר שצז]. }\\

\ערך{אב }\הגדרה{- המקים את הבית, המדריך את התולדות, המאיר את ארחות חייהם בהשפעתו הרוחנית }\מקור{[שם קיח]}\צהגדרה{. }\\

\ערך{הא }\הגדרה{- מורה על המוכן ומוכשר להושטה }\מקור{[עפ"י ר"מ פה]}\צהגדרה{. }\\

\ערך{האדרת שם ד'}\תקלה{ -}\הגדרה{ גלוי }\הגדרה{עז°}\הגדרה{ }\הגדרה{הגבורה־האלהית°}\הגדרה{ הפועלת את הכל למען הרוממות האצילית, המסוקרת אך לפני }\הגדרה{כסא־כבודו°}\הגדרה{ של בורא כל העולמים ברוך הוא. ההופעה העזיזה החודרת מרום הגובה העליון עד שפל המדרגה של אדם על }\הגדרה{הארץ°}\הגדרה{ }\מקור{[עפ"י ע"א ד ט קד]}\צהגדרה{. }\\

\ערך{האח }\הגדרה{- קריאת }\הגדרה{השמחה°}\הגדרה{ }\הגדרה{והחדוה°}\הגדרה{ }\מקור{[ר"מ קכ]}\צהגדרה{. }\\

\משנה{אהבת צור־העולמים°}\הגדרה{ - זיו }\הגדרה{השכינה°}\הגדרה{, הכרה שכלית והרגשית, ללכת }\הגדרה{בדרכי־ד'°}\הגדרה{ באהבת אמת והכרה עמוקה }\הגדרה{פנימית°}\הגדרה{ }\מקור{[עפ"י ע"א ג ב נ]}\צהגדרה{. }\\

\משנה{האהבה העליונה }\צהגדרה{- }\צהגדרה{אהבת־עולם°}\צהגדרה{ }\צהגדרה{ואהבה־רבה°}\צהגדרה{, אשר לישראל את ד' אלהיהם }\צהגדרה{ואביהם־שבשמים°}\צהגדרה{ מלך־עולמים, הבוחר בעמו ומלמדו ומדריכו }\צמקור{[ל"י א (מהדורת בית אל תשס"ב) צג]. }\\

\ערך{אור חֵי־העולמים° }\הגדרה{- }\הגדרה{הענין־האלהי°}\הגדרה{}\מקור{ [א' סו]}\צהגדרה{.}\\\הגדרה{}\הגדרה{הטוב־הכללי°}\הגדרה{, הטוב האלהי השורה }\הגדרה{בעולמות°}\הגדרה{ כולם. נשמת־כל, האצילית, }\הגדרה{בהודה°}\הגדרה{ }\הגדרה{וקדושתה°}\הגדרה{ }\מקור{[עפ"י א"ש פרק ב]}\צהגדרה{. }\\

\ערך{אור ד' המהוה הישות }\הגדרה{- }\הגדרה{זרוע־ד'°}\הגדרה{ אשר נגלתה, יסוד ההשתלמות הבלתי פוסקת }\מקור{[עפ"י א"ק ב תקל]}\צהגדרה{. }\\\משנה{אור ד' וכבודו }\הגדרה{- אמיתת }\הגדרה{הרצון־הכללי°}\הגדרה{ אשר }\הגדרה{בנשמת°}\הגדרה{ היקום כולו }\מקור{[שם ג לט]}\צהגדרה{. }\\\משנה{אור ד' }\הגדרה{- המגמה האלהית היותר ברורה ותהומית לאין חקר }\מקור{[פנק' ג שלא]}\צהגדרה{.}\\\הגדרה{}\הגדרה{נשמת־העולמים°}\הגדרה{ }\תקלה{[קבצ' ב קנה}\משנה{]}\צהגדרה{.}\\

\ערך{אב }\הגדרה{- הרועה הנאמן. מדריך, העומד במעלות נפשו הרבה יותר גבוה מהמעלה של הצעירות של }\הגדרה{הבנים°}\הגדרה{, הצריכה לקבל את }\הגדרה{השפעתו°}\הגדרה{ }\מקור{[עפ"י ע"ר ב סה]}\צהגדרה{.}\\

\ערך{אבנט}\הגדרה{ - מכוון בתור אמצעי, בין החלק העליון מקום הכחות הנפשיים, לבין החלק התחתון שבגוף, מקום הכחות הגופניים השפלים, שמורה אמנם על היחש החזק שיש לכחות השפלים אל הכחות הנפשיים, עד }\הגדרה{שהקדושה°}\הגדרה{ המעלה את הנטיות הנפשיות, פועלת להגביל יפה את סדרי הפעולות הטבעיות לצד המעלה והקדושה}\תקלה{ [}\הגדרה{ע}\תקלה{"}\הגדרה{א ג ב ד}\תקלה{]}\הגדרה{.}\\\הגדרה{ע"ע חגורה, יסוד הויתה.}\\

\משנה{עבודת ד' וכל מעגל טוב מאהבה }\הגדרה{- מידיעת }\הגדרה{הטוב°}\הגדרה{ הגנוז בהם }\מקור{[עפ"י ע"ר א רפו]}\צהגדרה{. }\\

\ערך{חסיה }\הגדרה{- }\משנה{"לחסות תחת כנפי־השכינה°"}\footnote{ מדרש תהילים קיז.}\הגדרה{ - שע"י }\הגדרה{הקדושה°}\הגדרה{ הנמשכת עליו }\הגדרה{מקבלת־עול־מלכות־שמים°}\הגדרה{ מתחדשת בקרבו רוח חדשה להיות למגן לו מכל המוני הדעות הרעות }\מקור{[ע"א ג ב קנא]}\צהגדרה{. }\\

\ערך{אהבה }\הגדרה{- }\משנה{שמרי האהבה }\הגדרה{- ע"ע תאוות. }\\

\משנה{שלימות האהבה}\הגדרה{ - השמחה הגמורה ואור הנפש, שעמה כל טוב ואושר ובה כלולים נועם החכמה וההשגה ואהבתה }\מקור{[ע"א א ד לו]}\צהגדרה{.}\\\מעוין{◊}\הגדרה{ }\הגדרה{האמונה°}\הגדרה{ והאהבה הן עצם החיים בעוה"ז }\הגדרה{ובעוה"ב°}\הגדרה{ }\מקור{[א' סט]}\צהגדרה{. }\\

\משנה{באהבה}\תקלה{ - }\הגדרה{בדרך חפץ פנימי והכרה עצמית }\מקור{[ל"ה 55]}\צהגדרה{. }\\\משנה{כח העבודה מאהבה}\הגדרה{ - }\מעוין{◊ }\הגדרה{אינו בא כי אם לפי מדת הידיעה הבאה בלימוד של קביעות ועשירות רבה במקצעות השונים של תורת }\הגדרה{המוסר°}\הגדרה{ }\הגדרה{והיראה°}\הגדרה{, שאי אפשר כלל להמצא מבלעדי לימוד בסדר נכון, למגרס תחילה בבקיאות מלמטה למעלה, ואחר כך למסבר בעומק עיון ודעה שלמה }\מקור{[ל"ה 188]}\צהגדרה{.}\\\הגדרה{ע"ע עבודה מאהבה, עבודת ד' מאהבה ותלמוד תורה־לשמה.}\\

\ערך{אמר }\הגדרה{לו }\תקלה{הקדוש ברוך הוא }\הגדרה{}\הגדרה{(לגבריאל°}\הגדרה{ שבקש להציל את }\הגדרה{אברהם־אבינו°}\הגדרה{ מכבשן האש) }\תקלה{אני יחיד בעולמי והוא יחיד בעולמו, נאה ליחיד להציל את היחיד}\footnote{ פסחים קיח:}\משנה{ }\הגדרה{- אברהם כדאי הוא בזכותו שיושפע עליו זה השפע של השווית רצונו עם רצוני, שישכיל בכל כוחותיו, ויושפע זה גם על נפשו המרגשת והצומחת, לדעת שאין שום דבר ראוי לרצון כ"א רצונו של הקב"ה, וממילא הכל מתייחדים בהשווי' אחת, ואין כאן ניגוד - }\משנה{נאה ליחיד להציל את היחיד }\מקור{[עפ"י פנק' ג ריג]}\צהגדרה{.}\\\הגדרה{ע' במדור מלאכים ושדים, גבריאל מציל מן הכבשן. ושם, יורקמו שר הברד מצנן את הכבשן להציל לצדיקים.}\\

\ערך{אתרא דיראה־עילאה°}\footnote{ אתרא דיראה עילאה, השלמות האלהית הנוראה - ע' זוהר ח"ב עט.}\הגדרה{ - }\הגדרה{השגוב°}\הגדרה{ }\הגדרה{העליון°}\הגדרה{ של השלמות האלהית הנוראה, שהוא }\הגדרה{האידיאל°}\הגדרה{ של הבריאה, היסוד של ההויה כולה }\מקור{[עפ"י א"ק ב תקלב־ג]}\צהגדרה{.}\\

\end{multicols}
\subsubsection{ב}\replacePolythumb{ב}
\fancyhead[CO]{אות ב}
\begin{multicols}{2}


\ערך{""בבגדו בה" בגין דבגדו בעבודה זרה"}\footnote{ ת"ז קמו: (קסח. במהדורה עם ביאור הגר"א) "ישראל וכו', אף על גב דהוו מחוייבין בכמה חובין, כמה דאמרין "אין בן דוד בא עד שיהיה דור שכולו זכאי או כולו חייב", אתמר בהון "אם רעה בעיני אדוניה אשר לא יעדה", עם כל דא "והפדה" מגלותא, ו"לא ימשול למוכרה" בגלותא, "בבגדו בה", בגין דבגדו בעבודה זרה".}\תקלה{ }\הגדרה{- ההסרה }\הגדרה{מעבודה־זרה°}\הגדרה{ שבתכונה החפצה בחיי }\הגדרה{צדק°}\הגדרה{ }\הגדרה{ויושר°}\הגדרה{, משפט־אמת וטהרת־משפחה. הטוהר המוסרי האידיאלי, שהוא תולדת }\הגדרה{האורה°}\הגדרה{ }\הגדרה{האלהית°}\הגדרה{ }\הגדרה{שבנשמה°}\הגדרה{ הבאה לסדר את החיים כולם ע"פ מבט בהיר, בטהרת רגש }\הגדרה{הכרת־הטובה°}\הגדרה{ החברתית המעשית הנובעת ממנו; וברוממות להבת האהבה אל הטוב והיושר בעצמו }\צהגדרה{[עפ"י מ"ר }\צמקור{285}\צהגדרה{].}\\

\משנה{לימוד }\צהגדרה{- }\משנה{חמשת חלקי עסק הלימוד }\צהגדרה{- }\משנה{א. שיטה }\צהגדרה{- שיטה מחשבית, בלא הגבלה, בלא ערך קצוב, הכל לפי גודל הרעיון, לפי אומץ השכל ולפי חריפות הבינה, עם זיכוך כח הדמיון ועומק הרגש. והיא מקפלת בקרבה ענינים לאין חקר, דנה עליהם בסקירה מהירה, לא יאומן כי יסופר, מעלה פנינים מזהירים מקרקעות ימים, מגלה אוצרות חושך ומטמוני מסתרים.}\footnote{ ע"ע ש"ק, קובץ א קנא.}\\\משנה{ב. ריהטא }\צהגדרה{- מין גירסא במרוצה גדולה, ברפרוף על הענינים, כמה שאפשר לקלוט, רק שהרבה ענינים יעברו דרך הפה והמחשבה. ולפעמים מדלגים איזה תיבות וענינים וקולטים אותם דרך המחשבה, מעשרים בזה את הידיעה בעושר כמותי, ומעודדים את חיי הרוח לחפץ של גדלות ורוחב התפשטות.}\footnote{ ע"ע ש"ק, קובץ א תשכז.}\צהגדרה{ }\\\משנה{ג. גירסא }\צהגדרה{- היא כבר מוגבלת, לדעת את הפירוש הפשוט, בלא עיון ובירור, מ"מ הרצאת הענין באה בהגבלה, ובמהירות האפשרית. }\\\משנה{ד. לימוד }\צהגדרה{- הולך במתינות ומלבן את הענין בהגבלתו המקומית יפה. }\\

\end{multicols}
\chapter{מדרגות והערכות אישיותיות}\addPolythumb{מדרגות והערכות אישיותיות}
\fancyhead[CE,CO]{מדרגות והערכות אישיותיות}
\begin{multicols}{2}


\תקלה{הגדרות מבוא:}\\

\משנה{אישיות }\צהגדרה{- }\משנה{ניכרת מעלתה, גדולתה ורוממות ערכה }\צהגדרה{- באחדות הכוחות אשר בקרבה, במיעוט הסתירה והניגוד שביניהם, בהתאם קישורם, איחודם וכיוונם, במערכת רוחנית ומעשית אחת, של מהות חיונית אחת }\צמקור{[ל"י ב ז]. }\\\הגדרה{}\הגדרה{במחשבה°}\הגדרה{, ברצון }\הגדרה{טוב°}\הגדרה{ }\הגדרה{ובדעה°}\הגדרה{ ברורה }\מקור{[קובץ א רנא]}\צהגדרה{.}\\

\ערך{גודל הנפש }\הגדרה{- }\הגדרה{הקדושה°}\הגדרה{ }\הגדרה{והטהרה°}\הגדרה{ }\הגדרה{הפנימית°}\הגדרה{, אומץ }\הגדרה{הרצון°}\הגדרה{ }\הגדרה{ועז°}\הגדרה{ }\הגדרה{המחשבה°}\הגדרה{ }\מקור{[א"ק ג קכג]}\צהגדרה{.}\\

\משנה{חכם }\צהגדרה{-}\משנה{ ה"חכם" התנכ"י }\צהגדרה{- <הוא אינו במובן פרטי של איזו חכמה או איזו ידיעה, או סכום חכמות רבות או ידיעות רבות, אלא במובן כללי מצד כלליות }\צהגדרה{גדלות°}\צהגדרה{ האישיות>. }\צהגדרה{אדם־גדול°}\צהגדרה{, בעל }\צהגדרה{רוח°}\צהגדרה{ גדול, בעל }\צהגדרה{נשמה°}\צהגדרה{ גדולה, בעל }\צהגדרה{קדושה°}\צהגדרה{, בעל כלליות, בעל נצחיות, המתרומם ומתעלה מעל חיי יום־יום והגבלות }\צהגדרה{החמר°}\צהגדרה{ אל גדולת הרוח וכלליות החיים, }\צהגדרה{מחיי־שעה°}\צהגדרה{ }\צהגדרה{לחיי־עולם°}\צהגדרה{, בעל גדלות של הכרה והרצון כאחד, של המדע }\צהגדרה{והמוסר°}\צהגדרה{ כאחד ביחוד }\צמקור{[צ"צ א כ].}\\\הגדרה{ר' במדור זה, תלמידי חכמים. ר' במדור הכרה והשכלה והפכן, "חכמה" שבכתבי הקודש.}\\

\משנה{שלמותם של תלמידי חכמים בישראל}\צהגדרה{ - }\צמשנה{(נערכת) }\צהגדרה{- <לא רק במידת הכמות של ידיעותיהם ושל התועלת הרוחנית שהם מביאים בלמוד ובמעשה; אלא בעיקר> בערך האיכות של }\צהגדרה{סגולת°}\צהגדרה{ אישיותם העצמית, הנקבעת ע"י מעלת התורה }\צמקור{[ל"י ב (מהדורת בית אל תשס"ג) עא].}\\\משנה{בן תורה }\צהגדרה{-}\צמשנה{ מוגדר }\צהגדרה{-}\צהגדרה{ <לא לפי מדת הידיעות של התורה אלא> לפי הערך של הקשור הנפשי אל התורה }\צמקור{[עפ"י ל"י ב (מהדורת בית אל תשס"ז) קכד].}\\\משנה{תלמיד חכם}\צהגדרה{ - }\צמשנה{(מתבחן)}\צהגדרה{ - באופי הנפשי, בקישור לתורה, בשאיפת החיים בתכונת בינת הלב וכשרון הדעת }\צמקור{[עפ"י ל"י ב (מהדורת בית אל תשס"ז) קכב].}\\

\ערך{תורה }\הגדרה{- }\משנה{(לימודיה וידיעותיה, מדד איכותם) }\הגדרה{- עומק ההבנה וחריפות השימוש בהם לכל חפץ. גודל הרושם שפועלים על הלומד, לענין התכונה של }\הגדרה{המוסר°}\הגדרה{ המעשים הטובים }\הגדרה{ויראת־ד'־הטהורה°}\הגדרה{ }\מקור{[עפ"י ע"א ב ט שמד]}\צהגדרה{. }\\\הגדרה{ע' במדור תורה, "חמאה של תורה". }\\

