

\end{multicols}
\chapter{מילון הראיה}\addPolythumb{הראיה מילון}
\fancyhead[CE,CO]{מילון הראיה}
\begin{multicols}{2}


\end{multicols}
\subsubsection{א}\replacePolythumb{א}
\fancyhead[CO]{אות א}
\begin{multicols}{2}
\תקלה{פשוט }\הגדרה{- }\תקלה{}\תקלה{(רצון־פשוט°}\תקלה{ בלשון }\תקלה{חכמי־האמת°}\תקלה{)}\footnote{ דעת תבונות סי' קנד.}\צהגדרה{ }\הגדרה{-}\תקלה{ בלי שום הרכבה ושום הכרח [עפ"י נ"א ה 18]. }\\

\משנה{פר (ענינו בקרבנות) }\הגדרה{- כח המפעל}\\

\ערך{לי }\הגדרה{- }\משנה{("להקריב לי" וכד') }\הגדרה{- אל }\הגדרה{הרום־העליון°}\הגדרה{, }\הגדרה{לאור־פני־מלך־חיים°}\הגדרה{ }\מקור{[ע"ר א קכט]}\צהגדרה{.}\\

\משנה{"יורו" שנאמר ביחס ל"בין קדש לחֹל" לעומת "יוֹדִעֻם" שנאמר ביחס ל"בין טמא לטהור"}\footnote{ יחזקאל מד כג.}\משנה{ }\צהגדרה{- לפי שיש הבדל במצב}\\

\ערך{"מעשים" }\הגדרה{- ע' במדור מצוות, הלכות, מנהגים וטעמיהן, הגדרות מבוא, מצוות, מעשה המצוות.}\\

\ערך{עזי פנים}\footnote{ \textbf{אשר קומטו בלא עת} - חגיגה יג:־יד., ע"ע ר"מ עג, עה. ה' ריט. וע' חסד לאברהם, מעין רביעי, עין יעקב, נהר לב "עמי הארץ הקשים והעזים הם ניצוצות חצונים שפזרה לילית הרשעה בישראל וגם הם עזי פנים בעלי זרוע המצערים ת"ח ועמדו כזרים כנגדם", ושם מעין ששי עין גנים נהר ה, ו, ז. היכל הברכה, דברים דף קעד. אבן שלמה פי"א. לשב"ו, הקדו"ש, שער ו פרק ה, ד"ה אמנם דע. \newline
(ע"ע של"ה, תורה שבכתב, פרשת ואתחנן, ד"ה גם סוד גדול. ובליקוטי תורה לאריז"ל, פר' דברים "והזמזומים הן מבירור חכמה אל מחשב' כד"א אשר זמם ומהם הע"ר וכל מי שיש לו הרהור ומחשבות רעות הוא מהם". אם הבנים שמחה, הקדמה שניה "עיין עוד בדברי חיים בהשמטות לפ׳ ויקהל שכתב בזה״ל: אך הע״ר, כל חסדים דעבדי לגרמייהו עבדי. כנראה בעליל שהרבנים וחסידים והבע״ב שבדור המה בעוה״ר רובן מע״ר ורוצים לשרור על הצבור וכל מעשיהם רק לגרמי׳ לקבל כבוד וממון ולכן אין להתחבר עמהם, רק אם עובדים באמת שמוסרים נפשם לד׳ לא לקבל שום תועלת לעצמם, עכ״ד. והוא מבהיל כל רעיון. וגם רבינו איש אלקים החת״ס בחלק ו בשו״ת סי׳ נט כתב כיוצא בזה" וכו' ע"ש שסיים: "הם כתבו כך בדורם שהי׳ באמת עוד דור דעה, מה נאמר בדורנו, ואנן מה נענין בתרייהו". ע"ע שיחות הרצי"ה מועדים ב, שיחה לתשעה באב תשל"ג. מה"ה ח"א פרק ח, המדות - הכנה לקבלת תורה. ושם: "שנאה היא ריקבון נפשי, מחלה". וכן בי' מאמרות לרמ"ע, אכ"ח, חלק א סי' כב: "שדמו בעצמם, אחי יוסף, היותם מספיקים להשלים האות שלו בדגלים ובצבאות ה', ואחר כך יכלו להמיתו ולא ימשך מזה גרעון במחנה שכינה לפי דעתם, והיה זה במדת השלום שהחזיקו בו בינותם בלב שלם ובכחו זה שתפו שכינה עמהם וכו'. אבל כתיב: 'אין שלום לרשעים', שאין שנאת חנם ושלילת השלום מצויה אלא בנשמות פגומות, וכל שכן אם בשכבר כשלו איש באחיו וכו' ותני לא מתו וכו' אלא מזרעו של ערב רב וכו'". ע"ע בפרי הארץ לרמ"מ וויטבסק, מכתבים מהמחבר הק' מאה"ק למדינת רוסיא, סוף המכתב הראשון. ובהיכל הברכה, שמות דף רעו.: ושם, דברים דף כט.).\newline
\textbf{מאורות }\textbf{דתוהו}\textbf{ }- ע' גר"א על תיקונים מז"ח, כז. ד"ה דקב"ה וד"ה ודא, ושם כז: ד"ה השפחות וילדיהן. שער התשובה לרד"ב, מהד' מקור תשל"ב, עמ' 20. ע"ע ש"ק, קובץ ג רפג, רצח־ט.}\תקלה{ }\הגדרה{- הם העומדים בכל דור נגד יסוד התפשטותה של }\הגדרה{הקדושה־העליונה°}\הגדרה{ }\הגדרה{ודעת־ד'־בעולם°}\הגדרה{, הם הם הנפשות הבאות מאותם הדורות, שהיו ראויים להיות לפני }\הגדרה{מתן־תורה°}\הגדרה{, אשר קומטו בלא עת, כלומר, אלה שהם בכל מהותם נגד }\הגדרה{התורה°}\הגדרה{ והשפעת קדושתה בעולם. המשוללים מכל הכרה }\הגדרה{טהורה°}\הגדרה{ }\הגדרה{וקדושה°}\הגדרה{, ומכל עול קדוש וראוי }\צהגדרה{[ע"ר א פ־פא, מ"ר }\צמקור{84}\צהגדרה{]. }\\\הגדרה{מנגדי התורה, בכל דור ודור }\מקור{[ה' ריט]}\צהגדרה{. }\\

\משנה{קֹדש }\צהגדרה{- }\צהגדרה{האצילות°}\צהגדרה{, מקור כל ההויה }\צמקור{[ח"ר 34]. }\\

\משנה{"פשט ידיה וקבליה" }\הגדרה{- ע' שם. }\\\ערך{תשובה }\הגדרה{- }\משנה{"עולה של תשובה" }\הגדרה{- ע' במדור פסוקים ובטויי חז"ל, עולה של תשובה. }\\

\ערך{עץ החיים }\הגדרה{- מצב }\הגדרה{המנוחה°}\הגדרה{ הטבעית, (בה היה }\הגדרה{אדם־הראשון°}\הגדרה{, בה) היה ראוי לדעת רק את הטוב. הסכמת הדעת שראוי לדבק בהדרכה הטבעית}\footnote{ ע' במדור זה, זיהרא עילאה דאדם הראשון.}\הגדרה{ ולא לעורר יותר מדאי את תשוקותיו }\מקור{[עפ"י ל"ה 40]}\צהגדרה{.}\\

\ערך{אמונה }\הגדרה{- }\משנה{"אֹמֶן"}\footnote{ \textbf{אומן} - עפ"י הקדמת ר"י הארוך בר קלונימוס האשכנזי (המיוחסת לראב"ד) לס"י, הנתיב הג'.}\משנה{ }\הגדרה{- }\הגדרה{החפש°}\הגדרה{, }\הגדרה{אב°}\הגדרה{ }\הגדרה{האמונה־העליונה°}\הגדרה{, השכל המקודש שהוא יסוד }\הגדרה{החכמה°}\הגדרה{ הקדומה, שמכוחו האמונה נאצלת }\צהגדרה{[עפ"י א"א }\צמקור{128}\צהגדרה{, שם }\צמקור{17}\צהגדרה{, (}\צמקור{77}\צהגדרה{)]. }\\

\ערך{אֹֽמֶר }\הגדרה{- }\משנה{אֹמֶר ההויה כולה }\הגדרה{- רוחה הפנימי של ההויה בסודה הכלול בקרבה}\\

\ערך{תֹּם }\הגדרה{- }\משנה{תוכן הַתֹּם }\הגדרה{- }\הגדרה{המערכה°}\הגדרה{ העליונה, מערכת }\הגדרה{הקודש°}\הגדרה{, העולה ממעל לכל ערכים }\הגדרה{מוסריים°}\הגדרה{, התמימות השלמה, שאין בה דופי פיסוק וקיצוץ מכל }\הגדרה{ההופעות°}\הגדרה{, }\הגדרה{יושר°}\הגדרה{ }\הגדרה{השכל°}\הגדרה{, יושר הלב, יושר הרגש, יושר }\הגדרה{הרוח°}\הגדרה{, יושר הטבע, יושר הבשר, יושר ההופעה, יושר ההקשבה }\מקור{[א' כט]}\צהגדרה{.}\\

\ערך{שם טוב}\footnote{ אבות ב ז, ד יג.}\תקלה{ }\הגדרה{- }\משנה{תואר שם טוב }\הגדרה{- ההשלמה שפועל על זולתו כפי ערכו }\מקור{[ע"א א ב נג]}\צהגדרה{. }\\

\משנה{אז }\צהגדרה{- במאמרי הראיה }\צמקור{165: }\צהגדרה{"כי }\תקלה{אז, דוקא אז}\צהגדרה{, כאשר עוד הפעם כיום צאתינו ממצרים נעמוד על רגלי עצמנו בעצמה הנשמתית, להיות דורכים על במתי ארץ בגאון ד' צור ישראל, אז יראו כל העמים צדקנו, ומשפט חירותינו יגלה ויראה על פני כל מלא עולם". ובש"ק, קובץ א תפח: בעלי הפנימיות משתוממים הם בעת שהחיצוניות נושאת את דגלה ברמה בחיים. אבל עליהם לדעת, כי הפנימיות היא }\תקלה{אז, רק אז,}\צהגדרה{ מנצחת, כשהיא מוצאה לפניה עולם ערוך ומסודר, אברים בריאים, ולב אמיץ, חושים מפותחים, וסדרי יופי, נקיות וטהרה, מוסר נימוסי, ודרך ארץ, ומרץ, ואהבה לחיים ולעולם. }\תקלה{אז}\צהגדרה{ תוכל הפנימיות לשלוט על ממלכה מלאה אונים". וביחס לחידוש הסנהדרין, עיכובים שבזמננו, בא"ה ב לח: "ישיבה המאוחדת שתהי' מאוגדת ביחד מגדולי רבני אה"ק וגדולי רבני הגולה תהי' האספה הגדולה הזאת נקראת "הרבנות הכוללת", והמובן יהי' הרבנות של כל ישראל, הגוי כולו. ואם חפץ השי"ת בידינו יצליח והכינוס הזה ואיגודו יבואו על נכון, ופעולות לטובה בין לחיזוק מצב התורה בכל עניני הדת, בין לפתרונן של השאלות היותר גדולות וכלליות הקשורות בחיי האומה בארץ ובגולה, ובין לתיקון מצב הכלל ביחש החיצוני כלפי האומות, בקשר עם חיזוק ידים של מליצינו ואוהבינו שבהן, בין להפרת עצת רשעי עולם שונאינו ומקטריגינו, בין בנוגע לחיי ישראל בארץ ובין בנוגע לחייו בכל תפוצות הגולה - כשכל אלה הדרכים יעוטרו באיזה מדרגה של הצלחה וכבוד, }\תקלה{אז, רק אז,}\צהגדרה{ תוכל לעלות על הפרק גם כן שאלה זו של השבת שבותינו בדבר ערכה של הסמיכה ואפשרותה. ורק דוקא לאחר כל המעשים הגדולים אשר ייראו מקיבוצינו וסידורינו, כי }\תקלה{אז}\צהגדרה{ יוכלו הדברים להיות במדרגה "מידי דקיימא לשאלה"". אמנם, במקומות רבים בהם משתמש הרב ב}\תקלה{"אז"}\צהגדרה{, כוונתו }\תקלה{"דוקא אז"}\צהגדרה{. בשם מו"ר הרב צב"י טאו.}\\

\משנה{יפה שעה אחת בתשובה ומעשים טובים בעולם הזה, מכל חיי העולם הבא. ויפה שעה אחת של קורת רוח בעולם הבא, מכל חיי העולם הזה}\footnote{ אבות ד משנה יז. ע"ע א"ק ב תקסז (לז).}\הגדרה{ }\צהגדרה{- (שעה אחת של) חיים אידיאליים והגונים מקימי רצון הבורא ומגמת הבריאה, }\תקלה{יפה }\צהגדרה{היא }\תקלה{מכל חיי העולם־הבא }\צהגדרה{כשלעצמם. ו}\תקלה{קורת רוח}\צהגדרה{ מהחיוניות הפנימית האדירה, מהאור והנעם הגדול והנשגב הממלא את ה"טרקלין", }\תקלה{יפה מכל חיי העולם הזה}\צהגדרה{ כשלעצמם }\צמקור{[עפ"י ל"י ב (מהדורת בית אל תשס"ז) שמ].}\\

\משנה{קדושה }\צהגדרה{- }\צמשנה{מהותה }\צהגדרה{- החשיבות הרוחנית שלפני האדם, ובאה ומאירה לו מלמעלה ממנו }\צמקור{[צ"צ קסו].}\\

\ערך{"בני אדם שהן ערומין בדעת ומשימין עצמן כבהמה"}\footnote{ חולין ה:.}\משנה{ }\צהגדרה{- }\משנה{בני אדם שערומים° בדעת }\הגדרה{- הם }\הגדרה{מושגחים°}\הגדרה{ מאד בעצמם, מצד שלמותם הפרטית, }\משנה{ומשימים עצמם כבהמה }\הגדרה{- מבטלים את עצמם אל הכלל, כאילו לא היתה להם תכלית פרטית כלל }\מקור{[מ"ש קיח־ט]}\צהגדרה{.}\\\ערך{בני אדם }\הגדרה{הקונים בשכלם, ע"י יגיעתם, מדות קדושות, באופן שיהי' להם בזה קנין טבעי, עד שטבעם יטה אותם לטוב, והם דומים בזה ל}\תקלה{בהמה, }\הגדרה{שכל פעולתה היא מצד הטבע לבד }\מקור{[עפ"י ע"ר ב קמט]}\צהגדרה{.}\\\ערך{בני אדם }\הגדרה{המשוללים נטיות חמריות המשימים עצמם כ}\תקלה{בהמה, }\הגדרה{שתהי' להם תשוקה בהמית, כדי להדריכה להיות עבודתם לד' עבודת }\הגדרה{עבד°}\הגדרה{, במעלת }\הגדרה{הטבע°}\הגדרה{ העליונה שלמעלה }\הגדרה{מהנס°}\הגדרה{ }\מקור{[עפ"י ע"ר ב קמט]}\צהגדרה{.}\\\הגדרה{ע' בנספחות, מדור מחקרים, עבד לפני המלך לעומת שר לפני המלך.}\\

\ערך{בני אדם שקיימו את התורה כולה מאלף ועד תיו}\footnote{ שבת נה.}\הגדרה{ - }\צהגדרה{<האדם שכבר התעלה להכשר השכלתה של תורה, שבא למדה זו שיודע את קונו, ומשיג את החפץ העליון המבוקש בכללותה של תורה, והכח הגנוז המתפרט בפרטיה מאור העליון, הוא משתמש בכל עניני התורה המצות וכונותיהם כשימוש האותיות הכוללים כל הצירופים שבעולם, להביע בהם המון מחשבות ורגשות לאין תכלית>. }\הגדרה{הדבקים בתורה בעצמם ובבשרם, אצלם מאירה אורה של תורה אפילו באור פניהם החיצוני, אצלם עומדת התורה כולה במדרגת אותיות הכוללים כל המבטאים, והמבטאים נולדים מהם חדשים בכל עת ורגע, "כי עדותיך שיחה לי". סידורה של תורה וערכה נערכת אצלם כערך כלל האותיות, שהם אבות כל מה שמצורף ושעתיד להצרף, כך כל מה שנולד ושיולד לטובה, לאמת, ליושר ולחכמה, הכל נובע מאורה של תורה, והדרכת מצותיה "למימינים }\הגדרה{בה"°}\הגדרה{, ובכל עת מצטרפים הרעיונות הפזורים לדברים שלמים, המופיעים דברים מלאים לכל ענין וכל חפץ }\מקור{[ע"א ד ה לד]}\צהגדרה{. }\\\הגדרה{ע' במדור זה, "תורתם אוֹמנותם".}\\

\משנה{אהבה לעומת טוב }\צהגדרה{- הטוב הוא בגדר "בכח", כי הוא נקודת הטוב האידיאלית הפנימית, הטוב האוביקטיבי, }\תקלה{-}\צהגדרה{ בהבדל מן "מיטיב" שהוא ביחש לאחרים, שכבר יש עם מה להיטיב, }\תקלה{-}\צהגדרה{ והאהבה היא לעומת זה בעיקרה בגדר "בפעל", אם יש אהבה בהכרח יש נאהבים, משא"כ במציאות הטוב, אינו מוכרח שיהיו מוטבים, מקבלים, }\תקלה{-}\צהגדרה{ אלא ממציאות "מיטיב". ומלבד זה בעצם תכונתם ואפים שונים וחלוקים הם המושגים האלה }\צמקור{[צ"צ כג].}\\

\ערך{"מאן דנטיר ברית° אתקרי צדיק°"}\footnote{ זוהר ח"א נט., צד., קסב., קצד:, רכז:, רמז:. ח"ב כג.}\תקלה{ }\הגדרה{- כשבקשת העמדת המשפחה היא נאורה ומובנת באלהיותה, ועל כן מקימה בקרבו צביון של טהרת רעיון ומנוחת לב. וכשתכסיס החיים כולו נעשה מובן, מיושב ומיושר, על פי הזוהר של ההמתקה של הנועם האלהי, נעשים  המעשים כולם מיושרים בתכלית המוסר והטוהר }\מקור{[קבצ' ב קעו]}\צהגדרה{. }\\\משנה{צדיק דנטיר ברית }\הגדרה{- מי שבא עד למדה העליונה, שהחפץ האלהי הוא גדול ומכריע בקדושתו גם את הנטיה }\הגדרה{המינית°}\הגדרה{, <הכרעה זו באה, לא בדרך עקירת הטבע הרוחני והגופני, כי אם ברוממותה אל התעודה השכלית, המוארה באורה האלהית> }\מקור{[קובץ א שכט]}\צהגדרה{. }\\

\ערך{שכינה }\הגדרה{- "}\משנה{ענן השכינה"}\footnote{ תנחומא במדבר יב.}\הגדרה{ - }\הגדרה{הוד°}\הגדרה{ }\הגדרה{הכבוד־העליון°}\הגדרה{, המעולף בענני ערפל לרדת ללב בני האדם להחיותם חיי }\הגדרה{עד°}\הגדרה{ }\מקור{[ע"ר א פח]}\צהגדרה{. }\\

\ערך{אֶבֶן דִי לָא בִיְדַיִן}\footnote{ דניאל ב לד "חָזֵה הֲוַיְתָ עַד דִּי הִתְגְּזֶרֶת אֶבֶן דִּי לָא בִידַיִן וּמְחָת לְצַלְמָא וגו'". ע"ע פנק' ג קנ.}\תקלה{ }\הגדרה{- }\משנה{די לא בידין }\הגדרה{- }\מעוין{◊}\הגדרה{ אם היה רצון }\הגדרה{השי"ת°}\הגדרה{ לעכב את ישראל עד שישלימו הם בעצמם במעשיהם הטובים את הכשרם אל }\הגדרה{הגאולה°}\הגדרה{, היתה האבן נקראת }\משנה{אבן די בידין}\הגדרה{, שבידים שהיו ישראל עושים את תיקוני שלמותם, בזה }\משנה{ימחו לצלמא°}\הגדרה{, שהוא כח הרע והקלקול שבמציאות שאינו נותן מקום לגדולת ישראל. אבל כיון שרצון השי"ת הוא רק שיהיו ישראל מוכנים קצת ואז השי"ת ינטלם וינשאם אל קדושת המעלה הראויה להם אחרי }\הגדרה{שימורק°}\הגדרה{ }\הגדרה{עונם°}\הגדרה{ ויקבלו איזו שלמות להכנה של מעלתם, אע"פ שלא יהיו עדיין מוכנים לגמרי, א"כ תהיה השבירה של הצלם ע"י }\משנה{אבן די לא בידין }\הגדרה{שלא עשו ישראל בידיהם זאת המעלה }\מקור{[מ"ש רעז־ח]}\צהגדרה{.}\\

\ערך{זיו אלהי }\הגדרה{- }\משנה{מעוז הזיו האלהי שממעל לכל גבולי־עולמים }\הגדרה{- }\הגדרה{הגודל־העליון°}\הגדרה{ }\מקור{[ע"ר א כ, ועפ"י שם ב עד]}\צהגדרה{.}\\\הגדרה{}\הגדרה{שכינתא°}\הגדרה{}\מקור{ [א"ק ג יט]}\צהגדרה{. }\\\ערך{זיו}\footnote{ \textbf{זיו} - בדעת תבונות, עמ' עה (מהד' ר"ח פרידלנדר) כותב הרמח"ל כהקבלה לסדרי הבריאה של הקב"ה ודרכיה בבריאה עצמה: "יש דבר אחד נמצא מהתחברות הנשמה והגוף - הוא זיו הפנים... ואין הזיו הזה נמצא לא לנשמה בפני עצמה, ולא לגוף בפני עצמו, אבל הוא הדבר הנולד מחיבור הנשמה והגוף ביחד". ובדרך מצותיך נא. "זיו הנפש הוא ענין חיוני מתפשט בגוף שהוא מקבל חיות זה וחי ממנו. וכך מהותו ועצמותו המחוייב המציאות, שהוא לבדו הוא, ואין זולתו, והוא חיי החיים, כשימשיך ויאיר ממנו ויגלה הארה בבחי' זיו ענינה הוא המשכת חיים בבחי' א"ס וז"ש הודו על ארץ ושמים כו' ופי' הוד וזיו". וע' בנספחות, מדור מחקרים, אור, זיו, ברק. ואור, זוהר, זיו.}\ערך{ אור אלהים }\הגדרה{- }\משנה{זיו החיים }\הגדרה{- }\הגדרה{שכינת־אל°}\הגדרה{. }\הגדרה{אור־אלהים°}\הגדרה{, הממלא את }\הגדרה{העולמים°}\הגדרה{ כולם}\\

\ערך{זמן }\הגדרה{- }\משנה{(לעומת נצח°) }\הגדרה{- ההוה התדירי }\מקור{[ע"ר א סד]}\צהגדרה{. }\\

\ערך{גשמים}\footnote{ אבות ה משנה ה.}\ערך{, הבאים מיסוד המים }\הגדרה{- מורים על עצם ההויה הגופנית, שֶׂכֶל הגוף }\מקור{[ע"ר ב קעו]}\צהגדרה{. }\\

\מעוין{◊ }\משנה{תאר מלך}\הגדרה{ ראוי להאמר מצד הכרתינו את }\הגדרה{דעת־האלהים°}\הגדרה{ מצד הנבראים כפי ערכינו }\מקור{[ע"א ב ח ו]}\צהגדרה{. }\\

\הגדרה{ע"ע אור עליון. ע"ע אור ד'.}\footnote{ \textbf{אור}\textbf{ אלהי, אור אלהים, אור ד', אור עליון }- בין מושגים אלה התקשתי למצוא הבדל, מכל מקום חולקו ההגדרות למחלקות שונות על פי המונחים השונים.}\הגדרה{ ע' במדור שמות כינויים ותארים אלהיים, אלהי, המקור האלהי. }\\

\ערך{תורה}\תקלה{60 }\הגדרה{- }\משנה{בעצמיות מהות הפנימית, הצורתית של התורה}\footnote{ \textbf{עצמיות מהות הפנימי הצורתי של התורה }- ע"ע לשב"ו, ס' הדע"ה, ח"ב, דרוש ד, ענף יב, סי' יב.}\הגדרה{ - עצמיות טבע נשמתם של ישראל }\מקור{[ע"א ד ט עו]}\צהגדרה{.}\\

\משנה{תורה }\צהגדרה{- }\צמשנה{מהותה המיוחדת של תורה לעומת הנבואה° }\צהגדרה{- השראת }\צהגדרה{השכינה°}\צהגדרה{ הכוללת, הקבועה, המוחלטת, "ואתה פה עמד עמדי", לעומת }\תקלה{הנבואה }\צהגדרה{המכוונת אל עניני הדורות ותמורות מאורעותיהם "לפי צרך השעה, הדור והמעשה"}\footnote{ ע' רש"י ד"ה תורת משה חולין קלז.}\צהגדרה{ }\צמקור{[עפ"י ל"י א מז]. }\\

\תקלה{טבע }\הגדרה{- }\תקלה{(במובן המדעי הרגיל) }\צהגדרה{-}\תקלה{ חוקי ברזל, החוקיות }\תקלה{העיוורת°}\תקלה{ השולטת במציאות }\תקלה{(שהקדושה°}\תקלה{ לוחמת כנגדה) [עפ"י ב"ר שצז]. }\\\משנה{טבעיות }\צהגדרה{- היציבות המסודרת של התופעות }\צמקור{[ב"א 10]. }\\

\ערך{חילה }\הגדרה{- הידיעה במה שאין ראוי }\הגדרה{לצייר°}\הגדרה{, כמו חלל פנוי במשפט הכרתו }\הגדרה{מיראת־ד'°}\הגדרה{ }\מקור{[ע"ר א רנו]}\צהגדרה{. }\\

\ערך{זָכָר }\הגדרה{- היסוד העקרי שבתולדה והמשכת החיים, }\הגדרה{הסגולה°}\הגדרה{ המפעלית }\מקור{[עפ"י ע"ר א מ]}\צהגדרה{.}\\

\ערך{אור חַי־העולמים° }\הגדרה{- }\הגדרה{אור־עליון°}\הגדרה{, מקור מקורות, חיי החיים }\מקור{[קובץ ה צט]}\צהגדרה{.}\\

\מעוין{◊}\\

\ערך{אור הגלוי }\הגדרה{- }\משנה{האור הגלוי}\הגדרה{° - האור הנראה של }\הגדרה{אור־התורה°}\הגדרה{ וחכמת ישראל כולה }\הגדרה{בקדושתה°}\הגדרה{ }\הגדרה{וטהרתה°}\הגדרה{, בבינתה והכרתה, בכבודה וישרותה, בעושר סעיפיה בעומק הגיונותיה ובאומץ מגמותיה. תלמודה של תורה בכל הרחבתה והסתעפו(יו)תיה, בדעת וכשרון, ברגש חי וקדוש, וברצון אדיר }\הגדרה{וחסון°}\הגדרה{ לחיות את אותם החיים הטהורים והקדושים אשר האור המלא הזה מתאר אותם לפנינו. }\הגדרה{(אור־הקודש־החבוי°}\הגדרה{) בהיותו מתקרב מאד אל מושגינו, אל צרכינו הזמניים, ואל מאויינו הלאומיים }\מקור{[מא"ה ג (מהדורת תשס"ד) קכג, קכה]}\צהגדרה{. }\\\הגדרה{ע"ע אור קודש חבוי. }\\

\תקלה{◊◊◊}\\

\תקלה{פרצופים: }\\

\משנה{פרצופים}\צהגדרה{ - }\צמשנה{עיקר ענינם }\צהגדרה{- }\צהגדרה{החיות האלהית העולמית והאחדות הכוללת }\צמקור{[ק"ה רכ]. }\\

\end{multicols}
\chapter{מונחי קבלה ונסתר}\addPolythumb{קבלה}
\fancyhead[CE,CO]{מונחי קבלה ונסתר}
\begin{multicols}{2}


\תקלה{חילוני }\הגדרה{- }\תקלה{נשמה חילונית }\הגדרה{-}\תקלה{ זרם ההויה הבא מצד המשך ההויה שאנחנו קוראים }\תקלה{טבע°}\תקלה{, ההשגה של אוה"ע, הזרם הבא מהאמצעים }\תקלה{מהסיבות°}\תקלה{ }\תקלה{[נ"א ה 31].}\\\הגדרה{ע' במדור פסוקים ובטויי חז"ל, שכינה, גילוי שכינה.}\\

\תקלה{טבע }\הגדרה{-}\תקלה{ החוקיות הכוללת את האדם והעולם. חוקיות החורזת את כל סדרי המציאות }\צהגדרה{-}\תקלה{ האדם, הדורות, ההויה וההסתוריה, שמים וארץ. חוקיות המארגנת את כל הברואים והיצורים לחטיבה אחת [עפ"י ב"ר שצז]. }\\

\ערך{אב }\הגדרה{- המקים את הבית, המדריך את התולדות, המאיר את ארחות חייהם בהשפעתו הרוחנית }\מקור{[שם קיח]}\צהגדרה{. }\\

\ערך{הא }\הגדרה{- מורה על המוכן ומוכשר להושטה }\מקור{[עפ"י ר"מ פה]}\צהגדרה{. }\\

\ערך{האדרת שם ד'}\הגדרה{ - גלוי }\הגדרה{עז°}\הגדרה{ }\הגדרה{הגבורה־האלהית°}\הגדרה{ הפועלת את הכל למען הרוממות האצילית, המסוקרת אך לפני }\הגדרה{כסא־כבודו°}\הגדרה{ של בורא כל העולמים ברוך הוא. ההופעה העזיזה החודרת מרום הגובה העליון עד שפל המדרגה של אדם על }\הגדרה{הארץ°}\הגדרה{ }\מקור{[עפ"י ע"א ד ט קד]}\צהגדרה{. }\\

\ערך{האח }\הגדרה{- קריאת }\הגדרה{השמחה°}\הגדרה{ }\הגדרה{והחדוה°}\הגדרה{ }\מקור{[ר"מ קכ]}\צהגדרה{. }\\

\משנה{אהבת צור־העולמים°}\הגדרה{ - זיו }\הגדרה{השכינה°}\הגדרה{, הכרה שכלית והרגשית, ללכת }\הגדרה{בדרכי־ד'°}\הגדרה{ באהבת אמת והכרה עמוקה }\הגדרה{פנימית°}\הגדרה{ }\מקור{[עפ"י ע"א ג ב נ]}\צהגדרה{. }\\

\משנה{האהבה העליונה }\צהגדרה{- }\צהגדרה{אהבת־עולם°}\צהגדרה{ }\צהגדרה{ואהבה־רבה°}\צהגדרה{, אשר לישראל את ד' אלהיהם }\צהגדרה{ואביהם־שבשמים°}\צהגדרה{ מלך־עולמים, הבוחר בעמו ומלמדו ומדריכו }\צמקור{[ל"י א (מהדורת בית אל תשס"ב) צג]. }\\

\ערך{אור חֵי־העולמים° }\הגדרה{- }\הגדרה{הענין־האלהי°}\הגדרה{}\מקור{ [א' סו]}\צהגדרה{.}\\\הגדרה{}\הגדרה{הטוב־הכללי°}\הגדרה{, הטוב האלהי השורה }\הגדרה{בעולמות°}\הגדרה{ כולם. נשמת־כל, האצילית, }\הגדרה{בהודה°}\הגדרה{ }\הגדרה{וקדושתה°}\הגדרה{ }\מקור{[עפ"י א"ש פרק ב]}\צהגדרה{. }\\

\ערך{אור ד' המהוה הישות }\הגדרה{- }\הגדרה{זרוע־ד'°}\הגדרה{ אשר נגלתה, יסוד ההשתלמות הבלתי פוסקת }\מקור{[עפ"י א"ק ב תקל]}\צהגדרה{. }\\\משנה{אור ד' וכבודו }\הגדרה{- אמיתת }\הגדרה{הרצון־הכללי°}\הגדרה{ אשר }\הגדרה{בנשמת°}\הגדרה{ היקום כולו }\מקור{[שם ג לט]}\צהגדרה{. }\\\משנה{אור ד' }\הגדרה{- המגמה האלהית היותר ברורה ותהומית לאין חקר }\מקור{[פנק' ג שלא]}\צהגדרה{.}\\\הגדרה{}\הגדרה{נשמת־העולמים°}\הגדרה{ }\צהגדרה{[קבצ' ב קנה}\משנה{]}\צהגדרה{.}\\

\ערך{"תשובה קדמה לעולם" }\הגדרה{- ע' במדור פסוקים ובטויי חז"ל. ושם, קדמה לעולם.}\\

\ערך{אב }\הגדרה{- הרועה הנאמן. מדריך, העומד במעלות נפשו הרבה יותר גבוה מהמעלה של הצעירות של }\הגדרה{הבנים°}\הגדרה{, הצריכה לקבל את }\הגדרה{השפעתו°}\הגדרה{ }\מקור{[עפ"י ע"ר ב סה]}\צהגדרה{.}\\

\ערך{אבנט}\הגדרה{ - מכוון בתור אמצעי, בין החלק העליון מקום הכחות הנפשיים, לבין החלק התחתון שבגוף, מקום הכחות הגופניים השפלים, שמורה אמנם על היחש החזק שיש לכחות השפלים אל הכחות הנפשיים, עד }\הגדרה{שהקדושה°}\הגדרה{ המעלה את הנטיות הנפשיות, פועלת להגביל יפה את סדרי הפעולות הטבעיות לצד המעלה והקדושה}\מקור{ [ע"א ג ב ד]}\צהגדרה{.}\\\הגדרה{ע"ע חגורה, יסוד הויתה.}\\

\משנה{עבודת ד' וכל מעגל טוב מאהבה }\הגדרה{- מידיעת }\הגדרה{הטוב°}\הגדרה{ הגנוז בהם }\מקור{[עפ"י ע"ר א רפו]}\צהגדרה{. }\\

\ערך{חסיה }\הגדרה{- }\משנה{"לחסות תחת כנפי־השכינה°"}\footnote{ מדרש תהילים קיז.}\הגדרה{ - שע"י }\הגדרה{הקדושה°}\הגדרה{ הנמשכת עליו }\הגדרה{מקבלת־עול־מלכות־שמים°}\הגדרה{ מתחדשת בקרבו רוח חדשה להיות למגן לו מכל המוני הדעות הרעות }\מקור{[ע"א ג ב קנא]}\צהגדרה{. }\\

\ערך{אהבה }\הגדרה{- }\משנה{שמרי האהבה }\הגדרה{- ע"ע תאוות. }\\

\משנה{שלימות האהבה}\הגדרה{ - השמחה הגמורה ואור הנפש, שעמה כל טוב ואושר ובה כלולים נועם החכמה וההשגה ואהבתה }\מקור{[ע"א א ד לו]}\צהגדרה{.}\\\מעוין{◊}\הגדרה{ }\הגדרה{האמונה°}\הגדרה{ והאהבה הן עצם החיים בעוה"ז }\הגדרה{ובעוה"ב°}\הגדרה{ }\מקור{[א' סט]}\צהגדרה{. }\\

\משנה{באהבה}\הגדרה{ - בדרך חפץ פנימי והכרה עצמית }\מקור{[ל"ה 55]}\צהגדרה{. }\\\משנה{כח העבודה מאהבה}\הגדרה{ - }\מעוין{◊ }\הגדרה{אינו בא כי אם לפי מדת הידיעה הבאה בלימוד של קביעות ועשירות רבה במקצעות השונים של תורת }\הגדרה{המוסר°}\הגדרה{ }\הגדרה{והיראה°}\הגדרה{, שאי אפשר כלל להמצא מבלעדי לימוד בסדר נכון, למגרס תחילה בבקיאות מלמטה למעלה, ואחר כך למסבר בעומק עיון ודעה שלמה }\מקור{[ל"ה 188]}\צהגדרה{.}\\\הגדרה{ע"ע עבודה מאהבה, עבודת ד' מאהבה ותלמוד תורה־לשמה.}\\

\ערך{אמר }\הגדרה{לו }\תקלה{הקדוש ברוך הוא }\הגדרה{}\הגדרה{(לגבריאל°}\הגדרה{ שבקש להציל את }\הגדרה{אברהם־אבינו°}\הגדרה{ מכבשן האש) }\תקלה{אני יחיד בעולמי והוא יחיד בעולמו, נאה ליחיד להציל את היחיד}\footnote{ פסחים קיח:}\הגדרה{ - אברהם כדאי הוא בזכותו שיושפע עליו זה השפע של השווית רצונו עם רצוני, שישכיל בכל כוחותיו, ויושפע זה גם על נפשו המרגשת והצומחת, לדעת שאין שום דבר ראוי לרצון כ"א רצונו של הקב"ה, וממילא הכל מתייחדים בהשווי' אחת, ואין כאן ניגוד - }\משנה{נאה ליחיד להציל את היחיד }\מקור{[עפ"י פנק' ג ריג]}\צהגדרה{.}\\\הגדרה{ע' במדור מלאכים ושדים, גבריאל מציל מן הכבשן. ושם, יורקמו שר הברד מצנן את הכבשן להציל לצדיקים.}\\

\ערך{אתרא דיראה־עילאה°}\footnote{ \textbf{אתרא}\textbf{ }\textbf{דיראה}\textbf{ }\textbf{עילאה}\textbf{, השלמות }\textbf{האלהית}\textbf{ הנוראה} - ע' זוהר ח"ב עט.}\הגדרה{ - }\הגדרה{השגוב°}\הגדרה{ }\הגדרה{העליון°}\הגדרה{ של השלמות האלהית הנוראה, שהוא }\הגדרה{האידיאל°}\הגדרה{ של הבריאה, היסוד של ההויה כולה }\מקור{[עפ"י א"ק ב תקלב־ג]}\צהגדרה{.}\\

\end{multicols}
\subsubsection{ב}\replacePolythumb{ב}
\fancyhead[CO]{אות ב}
\begin{multicols}{2}


\ערך{""בבגדו בה" בגין דבגדו בעבודה זרה"}\footnote{ ת"ז קמו: (קסח. במהדורה עם ביאור הגר"א) "ישראל וכו', אף על גב דהוו מחוייבין בכמה חובין, כמה דאמרין "אין בן דוד בא עד שיהיה דור שכולו זכאי או כולו חייב", אתמר בהון "אם רעה בעיני אדוניה אשר לא יעדה", עם כל דא "והפדה" מגלותא, ו"לא ימשול למוכרה" בגלותא, "בבגדו בה", בגין דבגדו בעבודה זרה".}\תקלה{ }\הגדרה{- ההסרה }\הגדרה{מעבודה־זרה°}\הגדרה{ שבתכונה החפצה בחיי }\הגדרה{צדק°}\הגדרה{ }\הגדרה{ויושר°}\הגדרה{, משפט־אמת וטהרת־משפחה. הטוהר המוסרי האידיאלי, שהוא תולדת }\הגדרה{האורה°}\הגדרה{ }\הגדרה{האלהית°}\הגדרה{ }\הגדרה{שבנשמה°}\הגדרה{ הבאה לסדר את החיים כולם ע"פ מבט בהיר, בטהרת רגש }\הגדרה{הכרת־הטובה°}\הגדרה{ החברתית המעשית הנובעת ממנו; וברוממות להבת האהבה אל הטוב והיושר בעצמו }\צהגדרה{[עפ"י מ"ר }\צמקור{285}\צהגדרה{].}\\

\ערך{צדקה תרומם גוי, וחסד לאֻמים חטאת}\footnote{ משלי יד לד.}\הגדרה{ - בישראל עיקר הצדקה תכליתה רק לרומם את הגוי, לרומם את הנותנים ולזכותם בעשיית הטוב והחסד, א"כ הוא מצד המעלה ולא מצד החסרון. ובאוה"ע אין כונתם כ"א מצד המצוקה הטבעית המורגשת במה שרואה צערו של המצטער, א"כ הוא רק מצד החסרון. <}\משנה{חטאת}\הגדרה{ - לשון חסרון, כמו "ולא יחטיא"> }\מקור{[פנק' ג רסא-רסב]}\צהגדרה{.}\\

\משנה{לימוד }\צהגדרה{- }\צמשנה{חמשת חלקי עסק הלימוד }\צהגדרה{- }\צמשנה{א. שיטה }\צהגדרה{- שיטה מחשבית, בלא הגבלה, בלא ערך קצוב, הכל לפי גודל הרעיון, לפי אומץ השכל ולפי חריפות הבינה, עם זיכוך כח הדמיון ועומק הרגש. והיא מקפלת בקרבה ענינים לאין חקר, דנה עליהם בסקירה מהירה, לא יאומן כי יסופר, מעלה פנינים מזהירים מקרקעות ימים, מגלה אוצרות חושך ומטמוני מסתרים.}\footnote{ ע"ע ש"ק, קובץ א קנא.}\\\צמשנה{ב. ריהטא }\צהגדרה{- מין גירסא במרוצה גדולה, ברפרוף על הענינים, כמה שאפשר לקלוט, רק שהרבה ענינים יעברו דרך הפה והמחשבה. ולפעמים מדלגים איזה תיבות וענינים וקולטים אותם דרך המחשבה, מעשרים בזה את הידיעה בעושר כמותי, ומעודדים את חיי הרוח לחפץ של גדלות ורוחב התפשטות.}\footnote{ ע"ע ש"ק, קובץ א תשכז.}\צהגדרה{ }\\\צמשנה{ג. גירסא }\צהגדרה{- היא כבר מוגבלת, לדעת את הפירוש הפשוט, בלא עיון ובירור, מ"מ הרצאת הענין באה בהגבלה, ובמהירות האפשרית. }\\\צמשנה{ד. לימוד }\צהגדרה{- הולך במתינות ומלבן את הענין בהגבלתו המקומית יפה. }\\

\end{multicols}
\chapter{מדרגות והערכות אישיותיות}\addPolythumb{אישיות מדרגות}
\fancyhead[CE,CO]{מדרגות והערכות אישיותיות}
\begin{multicols}{2}


\תקלה{הגדרות מבוא:}\\

\משנה{אישיות }\צהגדרה{- }\צמשנה{ניכרת מעלתה, גדולתה ורוממות ערכה }\צהגדרה{- באחדות הכוחות אשר בקרבה, במיעוט הסתירה והניגוד שביניהם, בהתאם קישורם, איחודם וכיוונם, במערכת רוחנית ומעשית אחת, של מהות חיונית אחת }\צמקור{[ל"י ב ז]. }\\\הגדרה{}\הגדרה{במחשבה°}\הגדרה{, ברצון }\הגדרה{טוב°}\הגדרה{ }\הגדרה{ובדעה°}\הגדרה{ ברורה }\מקור{[קובץ א רנא]}\צהגדרה{.}\\

\ערך{גודל הנפש }\הגדרה{- }\הגדרה{הקדושה°}\הגדרה{ }\הגדרה{והטהרה°}\הגדרה{ }\הגדרה{הפנימית°}\הגדרה{, אומץ }\הגדרה{הרצון°}\הגדרה{ }\הגדרה{ועז°}\הגדרה{ }\הגדרה{המחשבה°}\הגדרה{ }\מקור{[א"ק ג קכג]}\צהגדרה{.}\\

\משנה{חכם }\צהגדרה{-}\צמשנה{ ה"חכם" התנכ"י }\צהגדרה{- <הוא אינו במובן פרטי של איזו חכמה או איזו ידיעה, או סכום חכמות רבות או ידיעות רבות, אלא במובן כללי מצד כלליות }\צהגדרה{גדלות°}\צהגדרה{ האישיות>. }\צהגדרה{אדם־גדול°}\צהגדרה{, בעל }\צהגדרה{רוח°}\צהגדרה{ גדול, בעל }\צהגדרה{נשמה°}\צהגדרה{ גדולה, בעל }\צהגדרה{קדושה°}\צהגדרה{, בעל כלליות, בעל נצחיות, המתרומם ומתעלה מעל חיי יום־יום והגבלות }\צהגדרה{החמר°}\צהגדרה{ אל גדולת הרוח וכלליות החיים, }\צהגדרה{מחיי־שעה°}\צהגדרה{ }\צהגדרה{לחיי־עולם°}\צהגדרה{, בעל גדלות של הכרה והרצון כאחד, של המדע }\צהגדרה{והמוסר°}\צהגדרה{ כאחד ביחוד }\צמקור{[צ"צ א כ].}\\\הגדרה{ר' במדור זה, תלמידי חכמים. ר' במדור הכרה והשכלה והפכן, "חכמה" שבכתבי הקודש.}\\

\משנה{שלמותם של תלמידי חכמים בישראל}\צהגדרה{ - }\צמשנה{(נערכת) }\צהגדרה{- <לא רק במידת הכמות של ידיעותיהם ושל התועלת הרוחנית שהם מביאים בלמוד ובמעשה; אלא בעיקר> בערך האיכות של }\צהגדרה{סגולת°}\צהגדרה{ אישיותם העצמית, הנקבעת ע"י מעלת התורה }\צמקור{[ל"י ב (מהדורת בית אל תשס"ג) עא].}\\\משנה{בן תורה }\צהגדרה{-}\צמשנה{ מוגדר }\צהגדרה{-}\צהגדרה{ <לא לפי מדת הידיעות של התורה אלא> לפי הערך של הקשור הנפשי אל התורה }\צמקור{[עפ"י ל"י ב (מהדורת בית אל תשס"ז) קכד].}\\\משנה{תלמיד חכם}\צהגדרה{ - }\צמשנה{(מתבחן)}\צהגדרה{ - באופי הנפשי, בקישור לתורה, בשאיפת החיים בתכונת בינת הלב וכשרון הדעת }\צמקור{[עפ"י ל"י ב (מהדורת בית אל תשס"ז) קכב].}\\

\ערך{תורה }\הגדרה{- }\משנה{(לימודיה וידיעותיה, מדד איכותם) }\הגדרה{- עומק ההבנה וחריפות השימוש בהם לכל חפץ. גודל הרושם שפועלים על הלומד, לענין התכונה של }\הגדרה{המוסר°}\הגדרה{ המעשים הטובים }\הגדרה{ויראת־ד'־הטהורה°}\הגדרה{ }\מקור{[עפ"י ע"א ב ט שמד]}\צהגדרה{. }\\\הגדרה{ע' במדור תורה, "חמאה של תורה". }\\

\end{multicols}
\subsection{מחקרים, באורים}
\fancyhead[CE,CO]{מחקרים, באורים}
\begin{multicols}{2}
\end{multicols}
\subsection{והשוואות להגדרות ולמושגים}
\fancyhead[CE,CO]{והשוואות להגדרות ולמושגים}
\begin{multicols}{2}
\משנה{ודאות באמיתותה של התורה }\צהגדרה{- גישות שונות מצוינות בכתבי הרב למקור ודאותנו באמיתותה של התורה ולתוקף חיובה עלינו:  }\\\צמשנה{ו.}\צהגדרה{ הדיוק בצפיית העתיד של התורה. בע"ה קנא: "אמנם איזה אומה תוכל לשאת עליה ברמה את הדגל של עבודת האלהים האידיאלית? וכו'. רק אומה כזאת, שיש לה תורה שקפלה את מהלכה על־פי התגלות דעת אלהים כזאת, של אספקלריא המאירה, של "פה אל פה אדבר בו", שהציבה את העתיד שלה בכל בהירותו, בכל דיוקיו היותר פרטיים, מראשית הויתה עד כל נפילותיה ושיבתה לתחיה באחרית הימים, ברום מעלה ואורה נפלאה".}\צהגדרה{ אמנם יתכן שענין סעיף זה הוא אחד עם סעיף א כאן.}\\\צמשנה{ז.}\צהגדרה{ המחוייבות שלנו לתורה נובעת מנאמנותנו למסורת דורות האומה (ע' במדור תורה, תורה שבעל פה ובהערה שם לד"ה יסודה של תורה שבע"פ). יתכן אמנם שענין סעיף זה הוא אחד עם סעיפים א ב כאן, אך }\צהגדרה{מ"מ נ"ל דיש לחלק עפ"י בירורו של מו"ר הרב יצחק שילת (בס' בין הכוזרי והרמב"ם, פרק סגולת ישראל) את החילוק בין הסברת הרמב"ם להסברת ריה"ל במושג 'סגולת ישראל'. שלשיטתו של הרמב"ם 'סגולת ישראל' היא הפלא האלהי היוצר סבירות סטטיסטית גבוהה יותר להופעת מידות נעלות ודעות אמיתיות בישראל מאשר באומות העולם; ולריה"ל בכוזרי, 'סגולת ישראל' באה לידי ביטוי במציאות שפע העניין האלהי המתגבש לכלל הראויות לשמיעת הדיבור האלהי שהיא עניין אחד עם מגמת ההוויה הכללית המתרכזת ומתגלה בהסתוריה הישראלית. קווי ההסברה השונים האלה הם הקווים השונים גם בין שני ההסברים בסוגיין, בין העולה מסעיף ב המתאים לקו ההסברתי של ריה"ל, לבין סעיף ז שיכול להתאים לאופי הסברת הרמב"ם. נופך מיוחד מעניק לאוריינטציה זו פרופ' ברוך קורצוויל בספרו במאבק על ערכי היהדות, בפרק: פלוראליזם האנורמליות כיסוד הקיום היהודי, ע"ש עמ' 206-213.}\\

\משנה{השערה }\צהגדרה{- מקור ביטוי זה בכתבי הרב בזוהר וירא קג. (וע' כתם פז לר"ש לביא, ח"א רמ: "שערים מענין השערה ואומד") בא"ק ג עב: "וההשערה העליונה בגיאות ד', בסוד נודע בשערים בעלה". ובא' צב "מסתגלת היא להקשבה עדינה, לצלילים עליונים, באים וטסים מעולמים גבוהים ונאצלים, אשר כל חד וחד מקבל מהם כפום שיעורא דיליה "נודע בשערים בעלה", ומכל השיעורים יחד יגלה הוד השלום וזיו האמת, יסוד העונג ומילוי החיים, המלאים עבודה וחפץ אידיאלי טהור". ההשערה או האומדנה, בא"ק ג קיט "עסוקים אנו }\תקלה{בהשערות ובאומדנות}\צהגדרה{", היא "דרך }\תקלה{האומד}\צהגדרה{ של המחשבה הרמה", כלשונו בע"ה קל, זוהי השערת־הלב־האמונית, של המחשבה־היסודית. ו}\צהגדרה{ע' }\צהגדרה{במדור}\צהגדרה{ הכרה והשכלה והפכן,}\צהגדרה{ "אובנתא דליבא". בא' קכד "עיקר האמונה היא בגדולת שלמות אין סוף. שכל מה שנכנס בתוך הלב הרי זה ניצוץ בטל לגמרי לגבי מה שראוי להיות }\תקלה{משוער}\צהגדרה{, ומה שראוי להיות }\תקלה{משוער}\צהגדרה{ אינו עולה כלל בסוג של ביטול לגבי מה שהוא באמת". ובא"ק ב תכח מוסבר שה}\תקלה{השערות}\צהגדרה{ הנשגבות יוצאות ממקור אורו של הרצון, שהוא הופעת הרצון בתכנית האמונה, הבאה מתוך הכרה פנימית נשגבה מאד, ושעל ידן (של ההשערות, בגודל עז רעננותן, במעמקי הנשמה היחידית, השואבת את לשד חייה על ידי צירוף הקיבוציות הכללית, שהיא מקבלת את שפעה מיסוד ההויה כולה, שחכמת אלהים שופכת עליה תמיד את רוחה, בחזון ובפועל) מתישב העולם הרוחני כולו, עם כל ודאותיו הנפלאות, העומדות ממעל לכל הודאיות שבעולם. וכנ"ל בערך אוביקטיבי סוביקטיבי אין לערב בין מושג קודש זה ובין הספקולטיביות במדע החול, לכ"ש שאין לבלבל כאן עם טומאות הספקנות. עוד העיר חיים וידל, להסכמות הראיה סי' ס, שהביא שם את דברי המורה נבוכים, ח"ב פל"ב, (ושם שם פל"ח) שכח המשער הוא המכשיר את הנביאים לקבלת הנבואה ומדויק הוא באומד שלו אף כאשר פונה הוא למושגים קונקרטיים ומוחשיים, הרבה יותר מהמדידה של המבט החיצוני. ועל פי זה אפשר להסביר שההשערה היא האינטואיציה הנשמתית. (ע"ע בענין כח המשער בס' הברית, ח"א מאמר יז, פרק יב). אמנם בא"א 98 משתמש הרב במובן החיצוני של המושג "זהו התפקיד של ההשכלה העולמית לכל צדדיה במחקרה ו}\תקלה{בהשערותיה הפילוסופיות}\צהגדרה{". ע' }\צהגדרה{במדור מונחי}\צהגדרה{ קבלה ונסתר, שער, "שערי אורה". ובס' הוד הקרח הנורא עמ' מה־מו הרחבתי בהסברת הענין.}\\

